\documentclass[25pt, a0paper, portrait]{tikzposter}

\usepackage{graphicx}
\usepackage{wrapfig}
\usepackage{ETbb}
\usepackage[fontsize=28pt]{fontsize}
% \usepackage{microtype}  % conflicts with hyperref and tikzposter, see https://tex.stackexchange.com/questions/481636/incompatibility-between-tikzposter-class-microtype-and-hyperref-package
\usepackage{hyperref}

% \title{The Journal of Open Source Software}
% \author{\emph{EditorsWarrick Ball}
% \date{\today}
% \institute{University of Birmingham}
% \usetheme{Simple} % Default, Rays, Basic, Simple, Envelope, Wave, Board, Autumn, and Desert
% \usecolorstyle{Denmark}
% \titlegraphic{logo_large.png}
\settitle{}

% https://tex.stackexchange.com/questions/254257/tikzposter-and-doi-package-conflict
\def\HyperFirstAtBeginDocument#1{#1}
\begin{document}

% Title block with title, author, logo, etc.
% \maketitle

\makeatletter
    \setlength{\TP@blocktop}{.485\textheight}
\makeatother

% https://github.com/openjournals/joss/blob/main/public/logo_large.jpg
% logo, QR code

\block{}{
  \begin{center}
    \includegraphics[width=0.75\textwidth]{joss-logo-transparent-crop.png} \\
    \textbf{Editorial board} \\
    Arfon Smith (Editor-in-Chief),
    Warrick Ball,
    Samuel Forbes,
    Daniel S.~Katz,
    Rachel Kurchin, \\
    Kevin M.~Moerman,
    Kyle Niemeyer,
    Kristen Thyng,
    Chris Vernon, and 95 editors
  \end{center}
}

\begin{columns}
\column{0.5} \block{About}{The \href{https://joss.theoj.org/}{Journal of Open Source Software (JOSS)}
  is an academic journal (ISSN 2475-9066) that publishes short
  articles describing open source software with a research
  application.  The review process includes checking that the
  software itself
  \href{https://joss.readthedocs.io/en/latest/review_checklist.html}{meets
    some modern standards}, including having documentation, tests
  and community guidelines.  This way, JOSS
  aims to give software creators a citable artefact
  through which their research contribution can be recognised,
  and to encourage them to use good software practice.}

\block{Publication statistics}{
  \vspace{-1.32em}
  \begin{center}
    \includegraphics[width=0.43\textwidth]{joss-papers-per-year.png}
  \end{center}
  \vspace{-1.32em}
  \begin{center}
    \emph{Most-cited articles}:
  \begin{tabular}{p{0.62\linewidth}@{\quad}cc}
    \textbf{Title} & \textbf{Date} & \textbf{Citations} \\
    \href{http://doi.org/10.21105/joss.01686}{Welcome to the Tidyverse} & 2019-11-21 & 15\,610 \\
    \href{http://doi.org/10.21105/joss.00861}{UMAP: Uniform Manifold Approximation and Projection} & 2018-09-02 & 7\,244 \\
    \href{http://doi.org/10.21105/joss.03021}{seaborn: statistical data visualization} & 2021-04-06 & 5\,012 \\
    \href{http://doi.org/10.21105/joss.03139}{performance: An R Package for Assessment, Comparison and Testing of Statistical Models} & 2021-04-21 & 3\,257 \\
    \href{http://doi.org/10.21105/joss.00772}{ggeffects: Tidy Data Frames of Marginal Effects from Regression Models} & 2018-06-29 & 1\,952 \\
  \end{tabular}
  \end{center}
}

\block{Get involved}{
  \href{https://joss.readthedocs.io/en/latest/submitting.html}{\textbf{Publish your code!}}
  JOSS is developer-friendly.  If you've already developed a significant research code
  with an open source licence, good documentation and automatic tests,
  we think it should only take an hour or two to prepare and submit your paper to JOSS. \\
  \\
  \href{https://reviewers.joss.theoj.org/join}{\textbf{Volunteer to review!}}
  JOSS reviews are used as a chance to help creators improve their software.
  Reviewers are encouraged to open issues and iteratively improve the software, rather
  than provide a few monolithic reviews. \\
  \\
  \href{https://blog.joss.theoj.org/2025/08/call-for-editors}{\textbf{Become an editor!}}
  We currently have a call open for new editors, who find reviewers and guide them through
  the process.  Read about our
  \href{https://joss.readthedocs.io/en/latest/expectations.html}{expectations for editors}
  and apply via the JOSS blog post.
}

\column{0.5}
\block{Publication workflow}{
  \begin{center}
    \includegraphics[width=0.42\textwidth]{JOSS-flowchart-updated.pdf}
  \end{center}
}

\block{Behind the scenes}
{JOSS streamlines its editorial processes by using software
  development tools.  Pre-review and review discussions happen in
  \href{https://github.com/openjournals/joss-reviews/issues/}{GitHub issues}.
  Many editorial steps are handled by the editorial
  bot: a \href{https://buffy.readthedocs.io/}{Ruby package}
  that interacts with GitHub via its API to do things
  like generate the pre-review and review issues, post checklists,
  recompile the article PDF on request, check reference DOIs, and more! \\
  \\
  Manuscripts are prepared in a flavour of Markdown and compiled to
  PDF via Pandoc.  JOSS's compilation process is available to authors
  as a GitHub action, so compliance can be checked automatically.
}

% \begin{subcolumns}
%   \subcolumn{0.45}
%   \block{JOSS homepage}{\begin{center}
%       \url{https://joss.theoj.org} \\
%       \includegraphics[width=0.15\textwidth]{joss-qr.png}
%   \end{center}}
%   \subcolumn{0.55}
  \block{About this poster}{
Presented by Warrick Ball at the \emph{Ninth Annual Conference for Research Software Engineering}
(RSECon25) at the University of Warwick, UK, 9-11 Sep 2025. Publication workflow diagram by Kyle Niemeyer. \\

\begin{wrapfigure}[5]{r}{10cm}\vskip-1em\includegraphics[width=\linewidth]{by.png}\end{wrapfigure}
Licensed under a Creative Commons Attribution 4.0 \\
International License: \\
\url{http://creativecommons.org/licenses/by/4.0/}
}
% \end{subcolumns}

% \block{Further information}
% {\begin{center}\begin{tabular}{p{20cm}|p{10cm}}
% %   & \textbf{JOSS homepage} \\
% %   & \url{https://joss.theoj.org} \\
%       Presented by Warrick Ball at the \emph{Ninth Annual Conference for Research Software Engineering}
%       (RSECon23) at the University of Warwick, 9-11 Sep 2025. \includegraphics[width=0.15\textwidth]{by.png} &
%       \textbf{JOSS homepage}
%        \\
% \end{tabular}\end{center}}
\end{columns}

% \begin{columns}

% % FIRST column
% \column{0.6}% Width set relative to text width

% \block{Large Column}{Text\\Text\\Text Text Text}
% \note{Note with default behavior}
% \note[targetoffsetx=12cm, targetoffsety=-1cm, angle=20, rotate=25]
% {Note \\ offset and rotated}

% % First column - second block
% \block{Block titles with enough text will automatically obey spacing requirements }
% {Text\\Text}

% % SECOND column
% \column{0.4}
% %Second column with first block’s top edge aligned with previous column’s top.

% % Second column - first block
% \block[titleleft]{Smaller Column}{Test}

% % Second column - second block
% \block[titlewidthscale=0.6, bodywidthscale=0.8]
% {Variable width title}{Block with smaller width.}

% % Second column - third block
% \block{}{Block with no title}

% % Second column - A collection of blocks in subcolumn environment
% \begin{subcolumns}
% \subcolumn{0.27} \block{1}{First block.} \block{2}{Second block}
% \subcolumn{0.4} \block{Sub-columns}{Sample subblocks\\Second subcolumn}
% \subcolumn{0.33} \block{4}{Fourth} \block{}{Final Subcolumn block}
% \end{subcolumns}
% \end{columns}
% \block[titleleft, titleoffsetx=2em, titleoffsety=1em, bodyoffsetx=2em,%
% bodyoffsety=-2cm, roundedcorners=10, linewidth=0mm, titlewidthscale=0.7,%
% bodywidthscale=0.9, bodyverticalshift=2cm, titleright]
% {Block outside of Columns}{Along with several options enabled}

\end{document}
